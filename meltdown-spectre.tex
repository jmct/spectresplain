\documentclass{article}

\title{Meltdown and Spectre Explanation}
\author{JMCT}
\date{January 4, 2018}

\begin{document}
\maketitle

\section{Introduction}

There are really two big issues that came to light starting on Tuesday evening.
Open Source developers submitted patches to the Linux OS that fixed a serious
security problem: but the problem was not public yet, so late on Tuesday and
early Wednesday\footnote{Even when I called you, JMCB.} it was not 100\%
certain what the problem was.

By the end of yesterday several independent security researches were able to
reverse engineer what the problem was \emph{only based on what the solution
was}. By the end of yesterday the cat was out of the bag and the original
security researches published the flaws they had found, confirming what people
had been guessing. Because it is useful to have names, the researchers who
discovered these flaws named them: Spectre and Meltdown.

There's a lot of noise about how serious these issues are, or how easy it is to
mitigate. I'll address some of that at the end. First, we need to understand
\emph{what} the problem is, and that requires a little background and some
high-level analogies.

\section{Background}

Modern CPUs are very complicated machines. For the past 25 years, the majority
of that complication has been towards one end: performance. When you start
reaching the limits of how small you can get, you start figuring out clever
methods to get faster. One of the main tools that CPU designers have used in in
the pursuit of performance has been \emph{speculation}. It's important to
understand what speculation is in order to understand the exploits that were
discovered/announced, yesterday.

\subsection*{A Wood Shop}

You own and operator a woodworking business our of an industrial wood shop. You
have two clients, A and B, and they both have custom designs. One of the
reasons your business is doing so well is because your clients trust that you
keep their designs safe and prevent other clients of yours from figuring out
their designs.

At some point you decide that you want your process to speed up, so that you
can handle more orders. So you higher an efficiency expert to come and look at
where your employees can save time. The efficiency expert immediately spots a
problem:\footnote{Note, the efficiency expert's `problem' is with regards to
speed, not any other concern.} one worker, Alice, has to wait for another
worker, Bob, to finish some cuts before she can know where she has to drill
some holes. While she's waiting for Bob to finish, she is just sitting there,
that's wasted time!  The efficiency expert notices that Alice doesn't
\emph{really} have to know what Bob is doing, she just has to know if Bob is
working on client A's design, or client B's design. Once she knows that, she
can start drilling the holes she needs to drill. However, Bob's work is very
complicated, so he can't communicate what he's doing until he is finished,
which is what causes Alice to wait.

\subsubsection*{Enter Speculation}

The efficiency expert has a clever idea: Why doesn't Alice \emph{speculate}
on what Bob is doing? That way she can get started before he's finished!

When the efficiency expert tells you this you scratch your head and wonder
``what happens if she guesses incorrectly?''

``Oh, that's no problem. When Alice speculates incorrectly just have her throw
away her current work and start on the correct work. When that happens you're
no worse off than you are now, and as long as she tends to get it right, you'll
be saving all sorts of time!''

Convinced of his argument you give Alice her new instructions; you tell her
that she should try to figure out what Bob is working on, if she thinks it's
client A's design, she should start that work, if she thinks it's client B's
design, she should do that. If she guessed wrong just discard the work and
start on the correct work.


\subsection*{Back to CPUs}

Now, as with any analogy, the above isn't perfect. However, it is a fairly
decent explanation of how speculation works inside a CPU. You have some parts
of the CPU take a \emph{guess} on what work they'll need to do, and they start
doing that work before it's really needed (hence the term speculation). You can
have some fairly clever methods of guessing pretty accurately, which saves you
all sorts of time.

It should be reinforced: \emph{almost all modern CPUs perform speculative
execution}.\footnote{There are modern CPUs that do not perform speculation, but
they are usually custom CPUs for \emph{very specific} applications where it is
necessary to be able to predict performance very accurately. Speculation
hinders that sort of analysis because the timing can vary wildly (depending on
whether the speculation was correct or not). So for some applications you would
rather have a slower processor that is easier to predict.} It is one of the
foundational techniques of the past 20 years for achieving high performance.

Speculative execution isn't inherently good or bad, but it is about
\emph{performance} first and foremost. Other concerns you may
have\footnote{Security, correctness, simplicity, etc.} are usually made more
difficult by this technique. Whether it's worthwhile depends on your
priorities. The emphasis of mainstream computing has been on performance above
all other considerations. In that context speculative execution makes sense.

\section{Spectre}

Spectre is the first of two issues announced. Spectre affects the majority of
mainstream processors, whether they are from Intel, AMD, or ARM.

\subsection*{Back to the Wood Shop}

Alice's speculation has been paying off wonderfully! The time between the order
being taken and completed has been reduced drastically. You're hoping that if
you higher the efficiency expert again they'll be able to find some other way to
speed up the process.

This time they observe the new workflow and spot a potential waste of time:
Whenever Alice was \emph{wrong} about which product Bob was working on she
spent a lot of time discarding her current work, getting a new piece of wood,
and starting the correct work. So the efficiency expert asks a potentially
na\"{i}ve question:

``Could Alice just fill in the holes she drilled and use the same piece of
wood? That would save her having to throw away her current piece of wood and
getting a new one.''

You have your engineers check that the structural integrity of the final product
would not be compromised by this change, and then Give Alice her new directions:

When Alice incorrectly guesses what product Bob is working on, she is to fill
in the holes she has already drilled with wood filler, and us the same wood on
the correct design.

This works great! You have even faster turnaround times now. This works great
and everyone loves it for \emph{10 years}\footnote{This is not an exaggeration,
it took 10 years for a serious flaw in this strategy to be (publicly) discovered}.

Then one day a designer at client A notices that one of the pieces of wood they
received from your wood shop has filled holes in it. Knowing how wood shops
save time, they wonder if they could reliably trigger a miss-speculation.

This designer starts submitting designs with the sole intent of \emph{confusing
Alice}. The designer doesn't actually care about the end product, the designer
just wants to see what patterns they can find in the filled holes that Alice
produces when she guesses wrong.

By submitting very specific designs the designer is able to perfectly recreate
client B's product, just by looking at the holes that were filled in the wood.

Obviously, this is a problem.


\subsection*{Back to CPUs}


Spectre works like the above. By submitting very specific instruction to the
CPU you are able to figure out what other processes on the computer have in
\emph{their memory}. Even though the CPU is meant to stop you from seeing what
other processes are up to.

As I mentioned earlier, Spectre affect modern Intel, AMD, and ARM CPUs. Which
means that arbitrary \emph{user} programs can read the memory from other
programs running on the same machine. This is a fundamental violation of the
security contract that modern CPUs claim.

\section{Meltdown}

If Spectre sounded bad, meet Meltdown. Meltdown is very similar to Spectre,
except it only affect Intel, and this is not a distinction that Intel wants.

It requires a tiny extension to our wood shop analogy.

\subsection*{Back in the Shop}

The workflow Bob and Alice have has been working great. However, you were
always a little worried about Alice's speculation being a waste of wood, that's
fine for the client's work, but not for the in-house work you sometimes have
Alice and Bob do. Because you use very special wood for your internal projects
you never want the situation where Alice started working on an internal project
and then had to fill in those holes and use the same wood for a client.

So you tell Alice that if she ever thinks Bob is working on a jig for the shop,
or some other internal project, she can't start working on her part until she
knows for sure. This way you're never using your internal stock for client's
work. It means that it might be a bit slower for internal project, but you
are okay with the waste/speed trade-off.

\subsection*{Back to CPUs}

Meltdown takes the problem from Specter and goes one step further: Arbitrary
\emph{user} programs can read the memory of {the operating system} running on
the same machine.

So in our analogy, Intel \emph{never told Alice to not speculate on internal
projects}. So now a malicious client can try to trick Alice into leaking
\emph{internal} designs. This is \emph{very} bad.

Luckily there is a fix! Just ensure that Alice \emph{actually discards the
incorrect work}. Even more fortunately, you can do this in software! The
problem is that by making Alice completely discard that work, you no longer
have the speed improvements that you gained from that technique.

\section{Conclusion}

Without trying to be apocalyptic, this is a worst-case scenario for computer
security. Yes, there are software patches that can \emph{mitigate} it, but...

\begin{enumerate}
  \item it causes a significant decrease in CPU performance
  \item that assumes that every one of the \emph{billions}
        of computer users will apply the software update as
        soon as possible
\end{enumerate}

The first point is not an issue for \emph{individuals}. Most people are not
maxing out the CPU constantly and can afford a performance hit, even if
non-trivial. However, it is a very big issue for \emph{industry}. Intel has had
a dominant market position almost entirely because of their performance
profile. It turns out that part of that performance benefit was at the cost of
a significant security problem that people did not know they were paying.

The second point is a little more dire: People (in aggregate) are \emph{very}
bad at updating their software. This update (Apple's, and Microsoft's will be
available early next week) is essential. Think of the Intel chips in kiosks and
ATMs that are unlikely to get updated. Think of the ARM chips in smart TVs
that will never get updated.

Now, that's just the software patch that works \emph{around} the problem. The
problem itself cannot be fixed until new architectures are developed. This will
not a `step' iteration from previous designs, the entire way that speculative
execution works will have to be re-thought.

\subsection*{Other Thoughts}

Before the gritty details of the problem were published, the community was able
to figure out that \emph{something} was serious just based on the open source
patches that were being proposed.

So, even with details on how the security problem manifested, individuals and
research groups were able to do the following in a 24 hour period:

\begin{enumerate}
  \item Figure out that speculative execution led to a security bug
  \item Develop a proof-of-concept program that read single bits from
        privileged memory on Intel chips
  \item Develop a website that when you load it in Chrome can read the
        memory of arbitrary user programs also running on your machine
  \item Produce a proof of concept that allows a malicious program on an Intel
        CPU to gain root access to the machine, giving the attacker full
        administrative access to the machine and its contents
\end{enumerate}

Each one of those built up from the previous one, showing the problem was real
enough that even those without full details could create significant attacks.
Imagine what Nation-States would be able to do.

The Spectre, is bad, but not worse-case. Meltdown, which only affects Intel
chips, is a literal worst-case security scenario.

I am not being apocalyptic: In general people will be fine, and Google, Amazon,
Apple, etc. will be able to ensure that their cloud systems are updated and
patched. But it is not an exaggeration to say that there is a fundamental
security flaw at the core of how modern CPUs are designed. I am certain there
will be a hardware fix for this, but the very fact that it requires the
re-architecture of Intel CPUs is very significant. It is very likely that it
will change the course of development at Intel, AMD, ARM, and others, and cost
a significant amount of money. Let's say it take 4 years for Intel to design a
new architecture (I don't know how long it really takes, but definitely on the
order of years), now they have to look at all of the designs they are working
on and \emph{go back} and redesign the ones that are affected.

Intel has also charged a very pretty premium on their CPUs because of their
performance (forget gaming, I'm thinking about data center computers), it turns
out that they do not have the performance advantage that they advertised once
you take into account the workaround for the security issue. Cloud providers
\emph{by definition} share the computing resource amongst several clients. No
one would rent time on Amazon Web Services if they thought that opened up their
data to anyone else renting time on Amazon's infrastructure!

\end{document}
